\documentclass[a4paper]{jsarticle}
\usepackage{otf}
\usepackage{geometry}
\geometry{left=25mm,right=25mm,top=30mm,bottom=30mm}

\begin{document}
\section{内的/外的}
\begin{itemize}
	\item 普通に「性質」と言われるのは概ね「外的性質」である。
	\item ある対象がその性質を持っていないと想像すると、その対象の同一性が損なわれ、それゆえその性質を持っていないと想像することができないような時、その性質は「内的性質」となる
	\item 「内的性質」(=「形式」)は普通の意味の「性質」ではなく。その対象に与えうる性質の範囲を示したものに他ならない
\end{itemize}

\end{document}