\documentclass[a4paper,11pt]{jsarticle}
%\usepackage{otf}
\usepackage{multicol}
\usepackage{ascmac}
\usepackage{geometry}
\usepackage{amsmath}
\geometry{left=25mm,right=25mm,top=30mm,bottom=30mm}


\title{『論理哲学論考』}
\author{担当 情報科学科B2 伊澤侑祐}
\date{2015/10/26}

\begin{document}
\maketitle

\begin{center}
文芸部 第一回読書会
\end{center}


\begin{quote}
本書は哲学の諸問題を扱っており、そして------私の信ずるところでは------それらの問題が我々の言語の論理に対する誤解から生じていることを示している。本書が全体として持つ意義は、おおむね次のように要約されよう。{\bf{およそ語られうることは明晰に語られうる。そして、論じ得ないことについては、ひとは沈黙せねばならない。}}

かくして、本書は思考に対して限界を引く。いや、むしろ、思考に対してではなく、思考されたことの表現に対してというべきだろう。というのも、思考に限界を引くにはわれわれはその限界の両側を思考できねばならない(それ故思考不可能なことを思考できるのでなければならない)からである。

従って限界は言語においてのみ引かれうる。そして限界の向こう側は、ただナンセンスなのである。(『論考』序文)
\end{quote}
\section{序}
『論理哲学論考』は「論理」「哲学」「論考」とあるくらいだから、論理の哲学に関する書物であると思う人はいるだろう。しかし、前掲した序文を読むとこれは「言語の限界」を画定することであるのが目的であると明言している。

つまり『論考』の本当の主題は「言語」と「言語の限界」であり、それ自身としての「論理」ではないのである。つまり『論考』とは、その大半を論理の考察に充てながらも、自分の本当の主題は言語である、と主張する書物なのである。結局、『論考』に登場する「論理」とは「思考の限界」に関する考察なのであり、その意味では『論考』は言語の限界に関する書物なのである。

\subsection{論理について}
ではここで、論理について整理しよう。「言語」をめぐるウィトゲンシュタインの思考は、ケンブリッジにおいて彼が没頭した「論理とは何か」という問いに始まるものである。「論理とは何か」という問いこそ彼の哲学的思考の原点であり、論理の本質を問うものである。我々は言語に関する哲学的考察をウィトゲンシュタインの思考から辿っていくにあたって、まず「論理」について知らなければならない。それは20世紀ケンブリッジにおいて「論理学」と呼ばれていたものが何なのかを知ることに他ならない。

\subsubsection{論理学革命とウィトゲンシュタイン}
20世紀に論理学が経験した革命によってゲーデルの不完全性定理やコンピューターが生まれたことはよく知られている。この革命の核心は、フレーゲによる記号言語(記号論理学)の考察と、それを用いた推論の計算化の発明である。


\section{本文}
\begin{quote}
世界は成立していることがらの総体である。(1)
\end{quote}
「成立していることがら」とは、たとえばウィトゲンシュタインはウィーン生まれであるとか、ウィトゲンシュタインは第一次世界大戦で志願兵として従軍していたとか、現実世界の事実のことである。

\begin{quote}
論理空間の中にある諸事実、それが世界である。(1.13)
\end{quote}

世界に対して、論理空間は現実には成立しなかったことも合わせ、それら成立したこと・しなかったことを共に持つようなものを「論理空間」とウィトゲンシュタインは呼ぶ。つまり対象を引数として代入したとき、論理の可能性を出力したものの総体だと考えられる。

「論理空間」は『論考』において最上級の概念である。考えて欲しい。序文で「思考の限界」を画定するとウィトゲンシュタインは言っている。他方、論理空間とは世界の在り方の可能性としてのすべてである。つまり、論理空間の限界こそ思考の限界なのである。

\begin{quote}
\subsection*{世界}
世界は事実の総体であり、ものの総体ではない(1.1) 

成立している事態(=事実)の総体が世界である。(2.04)

現実の全体が世界である。(2.063)

所事態の成立・不成立の全体が現実である。(2.06)

\subsection*{事実}
成立していることがら、すなわち事実とは、諸事態の成立である。(2)

\subsection*{事態}
事態とは諸対象(もの)の結合である。(2.01)

諸対象が事態において結合する仕方が事態の構造である。(2.031)
\subsection*{対象}
対象はすべての状況の可能性を含んでいる(2.014)

対象は単純である。(2.02)

対象が世界の実態を形作る(2.021)

事態のうちに現れる可能性が対象の形式である。(2.041)

対象の配列が事態を構成する。(2.0272)

不変なもの、存在し続けるもの、対象、これらは同一である。(2.027)
\end{quote}

この一連の文によれば「世界」は「事実」$\rightarrow$「事態」$\rightarrow$「対象」へと分解されていくということがわかるだろう。


事実と事態を区別しよう。「事実」とは現実に起こっていることがらに対して、「事態」とは起こりうることがらであり、必ずしも現実に起こっているものに限らない。つまり、事実------現実性、事態------可能性ということになる。別の言い方をすれば、事実とは私が世界において現実に出会っている所与であり、事態とは『論考』が行うような分析の結果として要請されるものである。

ここで今まで登場した単語を整理しよう。

\begin{itemize}
\item 世界:成立していることがらの総体 = 事実の総体 $\neq$ ものの総体
\item 事実:所事態の成立 = 論理空間の中にあるもの
\item 事態:諸対象(もの)の結合。互いに独立(2.061)
\item 対象:もの = 事態の構成要素

\end{itemize}

「対象」とは、「事態」を構成するから分析の結果要請されるものとして考えて差し支えないだろう。現実性から可能性へと幅を広げる時、事実を分解して新たな再結合の可能性へと備えなければならないが、その時に分解された事実が構成用をが「対象」である。ただウィトゲンシュタインは「対象」に対して具体的な例を与えていない。「世界」$\rightarrow$「事実」$\rightarrow$ 「事態」$\rightarrow$「対象」と分解されてゆくが、「対象」は何なのかというと「もの」という定義しか与えられていない。特に性質や関係も対象なのかというと意見が分かれる場合が多いそうだ。そこで「対象」について考えることをここでの議題にしたい。「対象」について考えることは同時に事態、事実、そして世界を考えることに他ならないからである。
\begin{quote}
Q. 「対象」とはいかにして規定されうるか。
\end{quote}
この問いに答える上で2.014以降をじっくりと読み合せよう。具体例を挙げながら考えると理解しやすいかもしれない。

\section{言葉の意味}
\subsection{外的性質/内的性質}
ある対象にとってその性質を持たないことが論理的に考えられない時、その性質はその対象にとって「内的」であると言われる。それに対し、その性質を持つと考えられるならば「外的」と言われる。<ウィトゲンシュタイン>を一つの対象とするなら<結婚したことがない>というのは外的性質、<数2>を対象とすれば<3より小さい>というのは、<3より小さくない2>を考えられないため外的性質になる。

\subsection{項/座}
数学における変数とそこに代入される数を一般的に考えたもの。変数が「座」であり代入される数が「項」である。空間においては、空間点が座、物体が項になる。

\subsection{論理形式}
ある対象の論理形式とは、その対象がどのような事態のうちに現れるか、その論理可能性の形式のこと。たとえば対象aが赤い色をしていたとする。<aは青い><aは黄色い>という事態も可能である。この時「対象aは色という論理形式を持つ」という。対象aは様々な色と結びつくことから、対象aそれ自体は「無色」(2.0232)と言われるのである。


\end{document}