\documentclass[11pt,a4paper]{jsarticle}
\usepackage{amsmath,amssymb,amsthm,bm,delarray,url}
\usepackage[dvipdfmx]{graphicx}

\usepackage{type1ec} % <-- New!
\usepackage[OT2,T1]{fontenc}
\usepackage[russian,english,french,japanese,german]{babel}
%以上、Babel埋め込み

%送り仮名(ルビ)\ruby{漢字}{かんじ}
    \newcommand{\ruby}[2]{%
    \leavevmode
    \setbox0=\hbox{#1}%
    \setbox1=\hbox{\tiny #2}%
    \ifdim\wd0>\wd1 \dimen0=\wd0 \else \dimen0=\wd1 \fi
    \hbox{%
    \kanjiskip=0pt plus 2fil
    \xkanjiskip=0pt plus 2fil
    \vbox{%
    \hbox to \dimen0{%
    \tiny \hfil#2\hfil}%
    \nointerlineskip
    \hbox to \dimen0{\mathstrut\hfil#1\hfil}}}} 

%\usepackage[dviout]{graphicx}
%\setlength{\partopsep}{2.0pt}


%\newcommand\nexteq{\end{minipage}\\\begin{minipage}[t]{0.8\linewidth}} 

%\setcounter{chapter}{2}

\begin{document}

%\allowdisplaybreaks[1] %display中の改ページを寛容にする

\title{東工大文芸部『論理哲学論考』読書会 第2回}
\author{たけうち うみ}
\date{2015年11月9日}

%\twocolumn[
\maketitle
%\hrulefill
%{
%\small
%\begin{quotation}

%\begin{flushright}
%……チャダエフ宛て プーシキンの手紙 1836年10月19日付 \\

%—ソ連映画『鏡』(アンドレイ・タルコフスキー,1975) 劇中引用
%\end{flushright}
%\end{quotation}}
%\hrulefill
%]

\section{はじめに}\label{jo}

今回の読書会の予定は以下のとおりです(範囲:2.1-2.225および3-3.05):
\begin{enumerate}
\item
(たけう{\bf ち})「はじめに」予定確認[5 min]
\item
(ち)前回までの読書会の振り返り、特に前回までの課題\ref{zenkadai}の確認[10]
\item
(各自)今回範囲の読み返し[15]
\item
(小グループ内)感想出し(ブレインストーミングののりで)[15]
\item
(グループごと)まとまった感想の発表[15,ここまでで半分くらいかな?]
\item
(ち、各自ツッコミ)本文を本資料\ref{hombun}に従って読む[15]
\item
(ち)今回考える課題(事前に用意した本資料\ref{konkadai}および感想から抽出)の発表[5]
\item
(小グループ内)課題について軽く議論[15]
\item
(グループごと)課題についての意見発表、話し合い[10]
\item
(ち)まとめ、持ち越す課題のかんたんな確認[5]
\item
解散 or ごはん?
\end{enumerate}%
確定ではないので意見があればどうぞ。

前回出た参考文献は巻末にリストとして挙げておきました。断りのない引用は内容、ページ番号ほかすべて本読書会の標準テキストである[1]に準じます。

本筋からずれるため本会では軽く流す(予定ですが、もしもっと突っ込みたいという場合は時間を取ります)箇所については*まーくを見出しのはじめにつけておきます。

\section{前回までの課題} \label{zenkadai}

前回までの課題を確認する。なお、本会では特に今回の部分に関わりそうな重要な部分を抜粋して読み上げる。解決への糸口となりそうな点(思いつき)については矢印の後に補足している。

\subsection{序}
\begin{enumerate}
\item
\begin{quote}おそらく本書は、ここに表されている思想……をすでに自ら考えたことのある人だけに理解されるだろう(p.9)
\end{quote}
「すべての導出はア・プリオリに成立している」(5.133,p.79)のだから、本書の内容も先験的に、つまり読む人によらず理解されうるべきである。それなのになぜウィトゲンシュタインは大多数の人に理解されることをあきらめているのか?$\rightarrow$1.「成立している」ことは「示される」ことであって、「語られる」ことではないから([4,p.73]「示されるものと語られるもの」参照)。

2.『論考』の宛名先は、基本的にウィトゲンシュタインの周辺(ラッセル、フレーゲら)のごく小さな哲学、論理学のサークルであるから。リテラシーのない相手に読まれることをそもそも意図していない、あるいはそれほど重視していない。
\end{enumerate}

\subsection{本文}\label{zenkadai2}
\begin{enumerate}
\item
\begin{quote}
2.011 | 事態の構成要素になりうることは、ものにとって本質的である。
\end{quote}
「本質的」とは?本質的という言葉は経験に依存している気がする。「ア・プリオリ」でないのでは?$\rightarrow$1.とりあえず原文を見てみよう:
\begin{quote}
Es ist dem Ding wesentlich, ...
\end{quote}
es が コンマ以後の内容を指す仮主語で it のようなもの、dem Ding は定冠詞付きの与格\footnote{ロシア語と同じ。伝統的なドイツ語教育では3格と言う。}で「もの der Ding にとって」の意、wesentlich が「本質的な;基本的な;重要な」を表す形容詞。ようするに essential である([6]の両英訳はそう訳しているし、手持ちの辞書でも英訳に掲げられている)。言い換えればものであるための必要条件が「事態の構成要素になりうること」であると考えてよいだろう。これならば経験に依存しているというより、単なる主張・言明である。

\item
\begin{quote}
2.012 | 論理においては何ひとつ偶然ではない……
\end{quote}
「偶然」などの用語が確率論を思わせる。彼の確率論観とは?$\rightarrow$1.のちの本文に確率論が出てくる。「測度」なんかの用語も出てくるがいまの意味ほど厳密ではない。Wikipedia「アンドレイ・コルモゴロフ」より引用:
\begin{quote}彼以前の確率論はラプラスによる「確率の解析的理論」に基づく古典的確率論が中心であったが、彼が「測度論に基づく確率論」「確率論の基礎概念(1933年)」で公理主義的確率論を立脚させ、現代確率論の始まりとなった。
\end{quote}
いっぽうで『論考』の完成は1918年、「コルモゴロフはギムナジウムを1920年に卒業した」とあるとおり、まだコルモゴロフは学生である。


\item \label{kennen}
\begin{quote}2.0123 | 私が対象を捉えるとき、私はまたそれが事態のうちに現れる全可能性をも捉える
\end{quote}
ここの「捉える」とは?可能性にすぎない、起こらないかもしれない「事態」に現れる対象について、目で見ることや耳で聴くことといった感覚器官に依存した「捉える」という動詞が結びついて良いのか?$\rightarrow$1.原文を引用しよう:
\begin{quote}
Wenn ich den Gegenstand kenne, so kenne ich auch s\"amtliche M\"oglichkeiten seines Vorkommens in Sachverhalten.
\end{quote}
副文、主文(英語で言う従属節と主節)共に「ich が kennen する」という骨格である。動詞 kennen は[2]では「よく知る」と訳される。関連語として wissen があるが、使い分けについて手持ちの辞書(アクセス独和辞典 第3版)を引用すると:
\begin{quote}
kennen は具体的なかかわり合いを通して体験的に知っていることを表す(例文略) \\
それに対し、wissen は見聞などを通して知識として知っていることを表す
\end{quote}
とある。似たようなことは[3,p.54]のあたりにも書いてある。[3]と前回の読書会を踏まえた私の理解を述べる。「分析の結果として要請」された「諸対象の結合」である「事態」について、結合の仕方(2.032によれば結合の仕方=「構造」)のとりうる可能性(2.033によれば構造の可能性=「形式」\footnote{「内的性質」のことでもある}、つまり対象Aと対象Bは結びつくけれどAとCは結びつかない、といったこと)を(いろいろ対象A,B,C...を当てはめて試してみて)体験して知ることを条件\footnote{wenn は英語の when と似ており「とき」と訳されるが、基本的に「条件」を表す接続詞である。[6]では2つの英訳ともに if と訳される。思い切って「必要十分条件」と解してしまおう。}として、「『対象』を体験して知った」と言えるのである。言い換えれば、{\bf 対象が現れる事態の形式を kennen することで、またすることによってのみ、その事態に現れる対象を kennen できる}、ということである。この「形式」「事態」「対象」にまつわる議論は、「像」についてもパラレルに議論できる。だからこの議論は今回の範囲を見ていく際にも参照することとなろう。

\item
\begin{quote}
2.0131 | 空間的対象は無限の空間のうちにあらねばならない。(空間点は対照を項とする座\footnote{前回の読書会資料、または訳注参照。いちおう再度書いておくと、変数にあたるモノが「座」であり代入される数にあたるモノが「項」である。数ではないのでこのような語を使う。}である。)……
\end{quote}
「空間 Raum」とは何か?数学的な「空間」つまり「集合に何らかの構造を入れたもの」であって、対象 Gegenstand を要素とするものなのか?それともわれわれの住む三次元(あるいは四次元?十一次元?)の物質的な「空間」のことなのか?「無限」は明晰な対象なのか?「無限の」は要素が無限個の空間なのか、それとも空間が無限にあるのか?$\rightarrow$1.
\begin{quote}
6.031 | 集合論は数学ではまったくよけいである。[1] \\
 クラスの理論は、数学ではまったく余計である。[2] \\
 Die Theorie der Klassen ist in der Mathematik ganz \"uberfl\"ussig.
\end{quote}
以上、6.031にウィトゲンシュタインの「集合論」観は言い切られている。ただここでわれわれの習う数学における「集合論」全体を否定されていると考えるのは語弊がありそうで、[2]では「クラスの理論」と慎重に訳されているとおり、実際には「無限集合を対象とみな」す「カントールのような実在論的態度\footnote{集合の集合、そのまた集合、「以下同様」という無限の手順を許す態度}を認めない」[3,p.172]という意味であろう。

2.上で述べたとおり、ウィトゲンシュタインは基本的には集合論をあまり用いない(もちろん、積極的に否定しているのは「クラスの理論」であっていわゆる集合の考え方ではないが)。そして集合については Klass なる語を使っているのだから、空間は物質的な空間とみなして良いだろう。2.0251「空間、時間、そして色(なんらかのいろをもつということ)は対象の形式である」とあるとおり、時間や色と同列に考えられる「形式」である。\ref{zenkadai}.\ref{kennen}.の kennen の議論を思い出せば、「対象Aという項に対して空間のうちの『空間点』を座として与えられるかどうか kennen することは、対象Aを kennen する必要条件である」といえよう。そうするとここでいう「無限」とは、単純に物質的な空間が無限に広がっているというイメージでいいのではないか(もちろん宇宙のサイズは有限であるとかそういう議論はできてしまうのだが……ウィトゲンシュタインの時代の宇宙観はもしかしたら違ったのかもしれない)。

3.ただし、空間 Raum は「物質的な空間」とは別の用い方もされる。2.013「いかなるものも、いわば可能な事態の空間のうちにある」というのは、「可能な事態」を「擬・(物質的な)空間化」した言い方であろう。また重要な概念である「論理空間 logische Raum」も空間 Raum の語を含む。[1]の訳注(5)のとおり、「可能性の総体」を「空間」と呼ぶと捉えてもかまわないのであろう。上の議論を敷衍すると、「対象Aという項に対して Raum のうちのある(可能な)要素を座として与えられるかどうか」が「事態の構造の可能性」「事態の形式」であるというような Raum が、より一般的な空間と言われているものであろう(これを書いているいま少し眠いのでここの厳密な対応は考えられていない。こんがらがっている気がする)。
\end{enumerate}

\subsection{全体}
\begin{enumerate}
\item
ウィトゲンシュタインの文体が気になる。長い文も短文もある$\rightarrow$1.原文を読んでみよう。

2.ウィトゲンシュタインの文体はF・ラムゼイによって「連続的な散文を書かずに、叙述に登場する諸命題の重要性を強調するため、番号を付した短い命題を書きます……」[4,p.12]と評されている。いま[4]を引用したが、この「[第1部]ウィトゲンシュタインのテキストの特徴と読み方」には彼のテキスト構造を「ゲノム的不連続構造」として表してある。これを参照されたい。

\item
述語論理について$\rightarrow$1.ウィトゲンシュタインはフレーゲの述語論理、その記法である記号論理学の成果を多く受け継ぎながらも、完全に受け入れていたわけではないようだ。それについては後の部分を読むことで明らかになるだろう。

2.しかし、この記号論理学の記法は「論理形式」を理解する上で不可欠とも思われる。[3]ではあくまでウィトゲンシュタインは日常言語が十分論理を記述しうることを言っているが、いっぽうで「言語の諸機能の内、とくに像としての使用に焦点を当てる」[3,p.61]と言っている。言語の像としての使用、つまり2.182「すべての像は論理像\ruby{でも}{丶 丶}ある」というように、論理形式を可能性として持つところの像としての使用に焦点を当てる限り、記号論理学の記法は常に適用できて便利であるからである。

\end{enumerate}
\section{本文} \label{hombun}
\subsection{2.1-2.225 「像」}

{\small
\subsubsection*{*ウィトゲンシュタインの人称代名詞}
\selectlanguage{german}
\begin{quote}
2.1 | Wir machen uns Bilder der Tatsachen.
\end{quote}

2.1-2.225にかけて、「像」という概念が導入され、この概念についての主張が展開されていく。ここであえて原文を引用したのは、2.1 自体の主張とは何かということを考えるためである。この文の主語は「われわれ」という1人称複数人称代名詞である。もちろんこれは訳者による恣意的なものではなく、ドイツ語原文でも Wir という1人称複数人称代名詞が用いられる。
この文における「われわれ」の意味は、『論考』全体における wir の使い方を見ていくと(見ていかなくてもわかるかもしれないが)、「これから私と読者のみなさんは「像」の概念を導入しますよ」という意味ではなく、「(「論理」をもつところの)人間であるわれわれは一般に像を作るものである」という意味であることがわかる。
さて、よく見てみると本文においてこの \frq{} wir \flq が用いられるのはこの箇所が二度目である。なお、初めて現れるのが2.06「(われわれはまた、……と呼び、……とも呼ぶ)」という括弧の内部においてであり、これは「わたしと読者のみなさん」といった感じの意味である。他の用法も見ていこう。

2.1の次が2.223「像の真偽を知るためには、われわれは像を現実と比較しなければならない」という一般的な動作の主体を表すもので、目的文。
これは少し微妙で、たとえば日本語なら「われわれは」を除いて「……知ろうとするならば、その存在は」というように書いてしまうことができる。つまり目的文や条件文の場合は、「人間であるわれわれは」という意味を含みつつも、もう少しその主体性が弱められて「一般に」と言っているようにも感じられる。このような場合、「一般に」パターンとして扱う。

以下続けてみていくと3.001、ここでは2.1同様「像」を「作り machen」うることについて述べている。すなわち「人間であるわれわれは」という意味でよいだろう。

3.03「……というのであれば、そのときわれわれは……ならなくなる , weil wir sonst ... m\"usten」という「副文の内容が結論的には事実に反する」ために接続法第2式動詞 m\"uste による非現実話法を用いた理由の従属節で、原文では「そうなのであれば」という節が省略されていると考えられる。この節が前の節([1][2]の邦訳では原文のコンマで区切られた「節」は、よりわかりやすく2つの「文」に分けられている)「非論理的なものなど、考えることはできない Wir k\"onnen nichts Unlogisches denken, ...」の理由となっている。
これは「一般に」とも思えるが「人間であるわれわれ」が「非論理的なもの」を考えられないというように、「論理」を持ってしては「非論理的なもの」を考えられないということを強調している感じもする。\footnote{[2]では主文は「私たちは非論理的なことを考えることができない。」と、「私たち」がしっかり訳されている。このあたりに[1]と[2]の訳の方針の違いを見るような気もする}


3.031「……それがどのようであるかなど、われわれには\ruby{語}{丶}\ruby{り}{丶}えない Wir k\"onnen ... nicht {\it sagen}」で、ここは微妙なのだが「一般に」といった意味にも思えるし、「人間であるわれわれ」を強調して言うようにも思える。%なお、3.031の出だしは「かつてひとはこう言った。 Man sagte einmal, 」で、不定人称代名詞 Man が使われている。

3.0321「私たちは、……描写することができるけれど、 Wohl k\"onnen wir ... darstellen,」。ここは[1]では直接 wir が訳されていないため、邦訳は[2]を引用した。これは「一般に」パターンのように思われる。

3.05「ある考えが真であることを、かりに私たちがアプリオリに知ることができるとすれば、それは……その考えが……とわかる場合だけだろう Nur so k\"onnten wir a priori wissen, dass ein Gedanke wahr ist, wenn ... seine Wahrheit zu erkennen w\"are.」。これも[1]では「われわれ」などの語がないので[2]を引用した。ここも条件文であって「一般に」ということができる。以上で今回範囲までの \frq{} wir \flq の抽出は終わりである。\\


いっぽうで、「私」つまり1人称{\b 単数}人称代名詞 \frq{} ich \flq について見てみよう。基本的には wir と同じ3つの分類をおこなう。ich は2.0121「私が……考えることができるならば、そのとき私には……考えることはできない」で初めて現れる。これは条件文であり、「一般に」の意味(なのはわかるが、いったんここで動詞「考える denken」との結びつきに注目しよう。ここで wir が取られないのはなぜか、感覚的にはわかる。wir は「語る」「描写する」など、行為の結果が他の主体に明白になるような動詞と結びつく。「(像を)作る」も、作った結果は他の人間たちにもウィトゲンシュタインにも「私」(これを書いているわたし自身)にもあらわである。いっぽうで「考える」は自分の中で完結していて、基本的に他の主体がおらずとも(独我論をとって)独り我のみが存在すれば使える動詞である(wir とも結びついてしまっているのが問題ではあるが)。以下も動詞との結びつきに注目したい)。

次は2.0123「私が対象を捉えるとき、私はまた……捉える。」\footnote{前回問題の「捉える」である。\ref{zenkadai}で原文を挙げて検討したことを思い出そう。}、これも「とき」と訳されているが、原文では2.0121同様 Wenn にはじまる条件文であり、「一般に」の意味か。動詞は「捉える[1]/よく知る[2] kennen」。

以下、2.01231「対象を捉えるために、たしかに私は……捉える必要はない。しかし、……捉えなければならない」なる構文で、本質は条件文(で、いいのかな?)で、「一般に」の意味。これも kennen。

2.013「私は、……であると考えることはできるが、……ものを考えることはできない。」。これは「(「著者である」とも捉えうるが、「独り我のみが存在するところの「我」である」としたほうがよさそう。そのような)ウィトゲンシュタイン自身が」と捉えることもできなくはないが、やはり「一般に」の意味だろう。

2.02331「(識別の手がかりとなる性質が何もない)ときには、私はそのものを識別できない 〜, so kann ich ... nicht hervorheben」、soは「とすれば」のような意味だから条件文と捉えることができ、「一般に」パターンだろう。「識別する[1]/際立たせる[2] hervorheben/distinguish/distinguish[6]」、この動詞は[her こちらへ向かって][vor (こちらの)前へ][heben 上げる]から成る。「そのものを持ってきて抽出してやる」くらいの意味か。

そしてだいぶはなれて、今回の範囲からは脱するが3.12「われわれが思考を表現するのに用いている記号を、私は命題記号と呼ぶ as Zeichen, durch welches wir den Gedanken ausdr\"ucken, nenne ich das Satzzeichen.」。これはわかりやすくて、ich は「著者ウィトゲンシュタイン自身」が、(必然的な結びつき\footnote{ソシュールの記号論によれば「所記シニフィアンと能記シニフィエの結びつきは恣意的である」。「命題記号」と「呼ぶ」ということに「独り我のみが存在するところの「我」である」ことを強調する信念があるわけではない、単なる「著者」「考察者」としての ich。}はないが、この本を書く(あるいはもう少し本質的に、この考察をすすめる)上で)「呼ぶ」というところの1人称単数である。
それはこの文中の wir の用法と比較すればわかりやすい。ここでは「人間であるわれわれ」が「表現する ausdr\"ucken」であろう。

ウィトゲンシュタインにおける1人称複数と単数との使い分けについては、かれの独我論的な関心と直接関わる問題である。今回は曖昧に分類したが、もっと精密にこの使い分けを読んでいく必要があるだろう。
}

\subsubsection{「像」の導入とその立ち位置}
先の節で「像」という概念は単に導入されただけでなく、「(「論理」をもつところの)人間であるわれわれは一般に像を作るものである」というのが2.1以下の主張であることを確認した\footnote{めちゃくちゃ長くなってしまったが、ようするにここではそれが確認できればよかったのである}。
像とは、言ってしまえば言語の一般化である。像は、2.11「模型」とか、2.131「対象の代わり」として、2.11「諸事態の成立・不成立を表す」。成立・不成立を表せるから、成立する事態の集積である2.1「事実の像」も「作る」ことができる。
像の要素とものとの対応について、これらを座とする関係が2.1514「写像」関係である。対応付けるという点において、われわれの数学における「写像」概念\footnote{現代数学における写像の定義は、「もとの集合$A$の元$a$からある対応$\Gamma$によって定められる集合$B$の部分集合$\Gamma (a)$について、その部分集合$\Gamma (a)$の元がただ1つであるような対応$\Gamma$を写像という」であった。}に近しい。2.151は、「像の関係はものの関係の表現である」という意味で、写像関係であることの言い換えである。

少し注意しておくと2.141「像はひとつの事実である。」について、あっさりと述べられているが、一言加えて「像は{\bf それ自体}ひとつの事実である。」と強調する意図も読み取れよう。[3,p.41-]「2-4 言語がなければ可能性はひらけない」では、像は世界の「箱庭」と呼ばれている。その例として、像を持ってくるということは「音声」、「文字」、「インクの染み」、あるいは「紙切れ」という世界の具体的な事実として実在する現象や物体(モノゴト)を持ってくるということに対応している。音声の順番、インクの染みの並び方、紙切れの置かれ方と言った具体的なモノゴトの「形式」(そのうちにはインクの染みの色のような「色形式」もある)が、「対象」の「形式」を代理していて「写像形式」と呼ばれているのだ。「成立していない事態というのは、現実の代理物によって像として表現される以外、生存場所をもちえないのである。」[3,p.44]言い換えると、事実から作った像は常に事実である。それゆえに対象に還元することができ、その対象をまた「形式にそって」組み合わせて事実を作ることができる。しかし形式に沿ったとはいえ、一度分解した像の要素(それ自体は対象でもある)から改めて作った像(これはあくまで事実である)について、それのいわば「逆写像」としてつくられる「事実」のようなモノは、(ナンセンスではないが)真とも限らない可能性でしかない。この「事実」のようなモノはしかし、像と形式を共有しているから、事実にはなりうる=事態ではある。かくして成立していない事態が像の要素の「形式にそった」組み合わせ(の記号列)の逆写像として得られる。また成立していない事態とはこのようにしてしか表現されない。

さて、このようにして像を見てくると、この像についても「事実」であるから、「構造」「形式」などを定義していままでの「事態」同様に扱うことができる。そして2.16「事実は、像であるためには、写像されるものと何かを共有せねばならない」というのは、像の言い換えである。何か像の写像する事態と「形式」を共有するということである。
この像の定義から、2.182「すべての像は論理像\ruby{でも}{丶 丶}ある」=「すべての像の写像形式は論理形式でもある」が導かれる。
この論理形式に沿って描写された(描写するだけなら、2.22「描写内容の真偽とは独立に、その写像形式によって」行える。 描写されたものを「意味」と呼ぶ。)ある状況(それは「成立・不成立」を描きつつ「真偽」がまだ決まっていない。2.225「ア・プリオリに真である像は存在しない」のである)が2.222「真」であるか、あるいは「現実との一致」をみているかを知るには、現実と比較する他ない。つまり2.223「像の真偽を知るためには、われわれは像を現実と比較しなければならない」のである。

ここで強調するのは、2.21「像は現実と一致するかしないかである。すなわち、正しいか誤りかであり、真か偽かである」ということである。像においては「真か偽か」定まらないということはありえない、つまり数学で言うところの「命題」であるということで、その上で像の真偽はア・プリオリには決まらないということを言っている。さて、「形式」という語の理解はこの辺りの理解に非常に重要であるので、再度\ref{zenkadai2}.\ref{kennen}.kennen の議論を確認しておこう。


\subsection{3-3.05 「像」と「思考」}
\begin{quote}
3 | 事実の論理像が思考である。
\end{quote}
これから、3.001「「ある事態が思考可能である」とは、われわれがその事態の像を作りうるということにほかならない」の意味を考えよう。「事実の論理像」を1.12に従って言い換えれば、「何が成立しているのか/何が成立していないのか」=「事態の成立・非成立」の論理像のことである。だから、「事態が思考可能である」とは、「事態が成立しているかしていないかということの論理像を与えうる」と言い換えられる。このとき、2.182「すべての像は論理像でもある」ことからその事態の像はとうぜん作ることができる。

3.02「思考は、思考される状況が可能であることを含んでいる。思考しうることはまた可能なことでもある」。この命題を見るに、思考は「事態の論理像」と捉えても良さそうである。

さて、この3-3.05という部分はけっこう重要で、論理の本質を直接的に示している部分であるとも言われる([4,p.36]のA〜D分類参照)。「非論理的なものなど、考えることはできない」という言葉に集約される部分である。
論理的でないとは、論理形式に従わないということである。論理形式に従わないような「事態」に対して、像の描写との一致・不一致を確かめることはできない。このとき像は2.21「真か偽かである」ことはできない(ナンセンスであるとか、数学的な「命題」ではない、と表現できよう)。だから、そのような像を作ることはできないのである。
ここから、論理と像、とくに言語の限界の一致が示される。これは「序」の「したがって限界は言語においてのみ引かれうる」のひとつの有力な説明である。


\section{今回箇所の課題} \label{konkadai}
\begin{enumerate}
\item
3から3.001を導く際、「事態が成立しているかしていないかということの論理像を与えうる」という換言を行なった。
しかし、「事態の成立、不成立の可能性までを確定させた上で論理像を与えうる」、つまり「真偽の確定した論理像を作りうる」というのは、「われわれがその事態の像を作りうる」を導くには過剰な条件である。というのは、事態の像を作るだけならばその真偽は確定させなくてもよいからである。
ここの説明をうまくできないだろうか。$\rightarrow$1.ひとつには、わたしの行なった換言が間違っているという可能性はある。「事実」の解釈、「現実」との違いなどが引っかかる。
\item
上とほぼ同じだが、3.02から得られる「思考」の換言「事態の論理像」ではなくてなぜ3「事実の」論理像なのか?さらにいえば2.1も「事実の像」となっている。ここも「事態」でも良さそうな気もするが、なぜ?$\rightarrow$1.「事実」についてもういちどいろいろな記述を拾ってみよう。[2,xvi]では「事態が現実にそうなっていること」という。あるいは2.0121「あらゆる可能性は論理においては事実となる」など。

2.あるいはまた、3.031「「非論理的」な世界について、……われわれには\ruby{語}{丶}\ruby{り}{丶}えない」ということは、逆に言えば論理的な世界については、たとえ「偽」であっても語りうるということである。これはまた2.0121「あらゆる可能性は論理においては事実となる」の言い換えである。このような「論理」においては「事実となりうるもの」を、2.1や3では「事実」と言ってしまっているのではないか。
\item
3.05「ア・プリオリに正しい思考」とは何か。$\rightarrow$1.恒真命題、「AまたはAでない」など?(たぶんちがう。なぜちがう?)
\end{enumerate}


\section{補足} \label{hosoku}
読むために以下のような資料があると便利だと思ったこと:
\begin{enumerate}
\item
ウィトゲンシュタインによる番号付けを図示したもの。「2-2.01-2.011 (改行) $\vdash$ 2.012」みたいな感じで(うまく表現できない)。ウィトゲンシュタインによるこの「入れ子型」の番号付けの意義については[2,xiv]がわかりやすい。
\item
いままでの用語を本読書会の読みに従ってまとめたもの。毎回の資料作成者が前回までの読書会をふまえて書いてきて、ちまちま書き足していくことがのぞましい?
\item
「述語論理」についてまとめたもの。[2]の前文に当たる「高校生のための『論考』出前講義」(野家啓一)や[4,pp.43-57]がわかりやすいが、情報系でつかうような正確な文献もほしいかもしれない。
\end{enumerate}


%トロフィム・デニソヴィッチ・ルィセンコ\footnote{
%\selectlanguage{russian}
%Trofim Denisovi{ch} Rysenko%Трофим Денисович Лысенко
%,1898-1976.
%\selectlanguage{japanese}



\begin{thebibliography}
{9}
\bibitem{}ウィトゲンシュタイン『論理哲学論考』,野矢茂樹 訳,岩波書店,岩波文庫,2003年8月19日初版

\bibitem{}ヴィトゲンシュタイン『論理哲学論考』,岡沢静也 訳,光文社,光文社古典新訳文庫,2014年1月20日初版
\footnote{以上、文庫化している『論考』の邦訳である。これらは底本が異なっており、さらに訳出の方向性も異なっている。[1]は1933年改訂版(序文にあたる「バートランド・ラッセルによる解説」が付されているため、おそらく独英対訳版)が底本である。[2]のほうがよりあたらしい研究者向きの批判版を底本にし、ドイツ語の細部に忠実に(慎重に)訳しているという。}

\bibitem{}野矢茂樹『ウィトゲンシュタイン『論理哲学論考』を読む』,筑摩書房,ちくま学芸文庫,2006年4月10日初版

\bibitem{}鬼界彰夫『ウィトゲンシュタインはこう考えた』講談社,講談社現代新書,2003年7月20日初版

\bibitem{}永井均『ウィトゲンシュタイン入門』,筑摩書房,ちくま新書,2005年9月22日初版

\bibitem{}Ludwig Wittgenstein. {\it Tractatus Logico-Philosophicus / Logisch-philosophische Abhandlung }SIDE-BY-SIDE-BY-SIDE EDITION,VERSION 0.42(January 5, 2015), Available at: \url{http://people.umass.edu/klement/tlp/}
\footnote{ようするに、Web 上フリーに入手できる原著の独英英対訳版。非常にありがたい}


\end{thebibliography}


\end{document}