\documentclass[a4paper,onecolumn,openany,article]{jsarticle}
%\usepackage{otf}
\usepackage{algorithm}
%\usepackage[top=30truemm,bottom=30truemm,left=25truemm,right=25truemm]{geometry}

%\title{TokyoTech Literature Club Reading Session Part4}
%\author{Yusuke Izawa}
%\date{December 24, 2015}

\title{東工大文芸部『論理哲学論考』読書会 第4回}
\author{伊澤侑祐}
\date{\today}

\begin{document}
\maketitle

\section*{範囲}
『論理哲学論考』(ウィトゲンシュタイン(訳:野矢茂樹) 岩波文庫) 4.2 | 5.5423

\section{この読書会の進め方}
範囲が長いため、今回は最後の範囲までカバーしたレジュメを作ることができませんでした。したがって、今回の読書会は知識の確認とします。内容に対する議論は次回行いたいと思います。今回は次のような、短縮バージョンでやりたいと思います。
\begin{enumerate}
  \item 予定の確認
  \item 該当範囲の読み合せ(感想出しは今回はなし)
  \item 前回まで読書会の振り返り、課題の整理
  \item 内容確認
  \item レジュメ読み合せ
  \item 忘年会
\end{enumerate}

\section{前回の問いに対する応答}
\begin{itemize}
  \item 言語とは日常言語のことを言うのか?

  非日常的な言語を私たちは思考できない。そもそも、思考可能か不可能であるかの領域は言語によって引かれているはずであり、事実から作られる命題は知覚可能なものであるから、それは日常言語で構成されているはずである。
  \begin{quote}
     4.002 | 日常言語から言語の論理を直接に読み取ることは人間には不可能である。
  \end{quote}

  これは果たして日常言語からの脱却を目指していることを示唆しているのだろうか?ラッセルは「論考」が完全な理想言語を満たす条件を考察していると述べたが、これはラムジーによる指摘\footnote{野矢先生の本のp.134を参照}で誤りであることがわかっている。したがって、言語とは完全に日常言語と異なるものではなく、日常言語との格闘の末生み出されるもの、と解釈すべきだろう。

  \item 「分析可能」とは何を以ってそう言うのか?

  それは点灯空間の例を見れば明らかだろう。起こりうる状況を真/偽で過不足なく記述している。つまり「完全に記述できる」という状況を以って「分析可能」というのである。

  また、以降の内容で出てくるが、「分析可能」を「分解/構成」が可能であると捉えるのもいいかもしれない。

\end{itemize}

\section{本文}
本章は前章までと比べるとテクニカルな部分が多い。それは、真理操作というプロセスの説明が多く、基礎的な概念の説明は前章まででほぼ終えているからである。したがって、本レジュメも処理の解説に重きをおく。まずは、これまでの構図を簡単に振り返ろう。

まず、論考は現実世界と日常言語から出発した。日常言語を分析し、再び日常言語を構成するという往復運動を行い続け、思考可能性の全体を明確に見通すことを目指している。名や対象を分析するのではなく、理想言語を追求するのではない。あくまでも、始まりと終わりは日常言語にある。

\subsection{分析}

\begin{quote}

4.22 | 命題を分析していけば、その結果は明らかに、名が直接結合してできた各要素命題でなければならない。要素命題が存在するからこそ、いかにして命題と命題の結合がなされるものか問題になるのである。

4.23 | すべての真な命題要素の列挙によって、世界は完全に記述される。世界は、すべての要素命題をあげ、さらにどれ真でどれが偽かを付け加えれば、完全に記述しうる。


\end{quote}

本章のはじめには分析について書かれている。分析とは、分解と構成の繰り返しである。これを簡単に言うと「{\bf{バラして組み合わせる}}」ことである。バラし方にもルールがあって、それは論理形式に則っていなければならないということである。

また、ここにある「論理語」とは「かつ」「または」「よって」という接続詞であり、名ではない。これについては後ほど確認する。


\subsubsection{分解のプロセス}

\begin{itemize}

\item 第一段階:\\命題の検証や推論といった言語実践の中で、有意味/無意味/ナンセンスを弁別する我々の"言語直感"を頼りに分析がなされ、要素命題と論理語が区別される。

\item 第二段階:\\要素命題は名と対象の対に分解される。そして名は、論理形式によってどのような配列が可能であるかチェックされ、対象も論理形式によっていかなる事態の構成要素となるかチェックされた上で分解される。

\end{itemize}



\subsubsection{構成のプロセス}

\begin{itemize}
\item 第一段階:\\ 名の論理形式にしたがって可能な要素命題のすべてが構成される。構成された要素命題は、すべての可能な事態を表現するものとなる。
\item 第二段階:\\ 事態の集合として状況が作られ、可能な状況の全体として論理空間が貼られる。また、命題では論理語によって要素命題から複合命題が作られる。

\end{itemize}


\subsection{論理記号}
\begin{quote}
  4.0312 | 「論理定項」は何ら対象の代わりをするものでない
\end{quote}
 とあるように、記号$「\sim」「\vee」 「\, . \,」 「\supset」$は論理語(論理定項)といい、名ではなく、対象でもない。つまり「操作」である。
各操作の真偽は4.3のような真理値表に従う。

\begin{itemize}
  \item 否定 $\sim p$ |「pでない」
  \item 論理和 $p\vee q$ |「pまたはq」
  \item 論理積 $p.q$ | 「pかつq」 $p \wedge q$と同じ意味 \footnote{ちなみにHaskellでは関数合成をドット$.$で表現する。Haskellでは右から評価されるが、\$ でつなぐことによって左から評価されるようになる)} (カッコ()の代用として捉えると良い)
  \item 条件法 $p \supset q$「pならばq」
  \item $p \, | \, q$ | 「pでもqでもない」
  \item (x).fx |  「すべてはfである」または「すべてのxに対してxはfである」
  \item ($\exists$ x).fx | 「fであるものが存在する」または「あるxに対して、xはfである」
\end{itemize}

推論関係が包含関係で捉えられるのは、私たちにとっては非常に馴染みのある事実であろう。


\subsection{命題}

真理条件について知識を確認するために、本文を参照しよう。

\begin{quote}
  4.3 | 要素命題の真理可能性は、事態の成立・不成立の可能性を意味している。

  4.41 | 要素命題の真理可能性が命題の真偽の条件である。

  4.431 | 要素命題の真理可能性との一致及び不一致の表現が、それぞれその命題の真理条件を表している。
\end{quote}

つまりここで大事なのは「真理可能性との一致/不一致」である。また、真理条件の中には「トートロジー」「矛盾」という極端なものがある。

トートロジーは恒真命題、つまり全命題の内側の極限点、矛盾とは恒偽命題、つまり全命題の外側の極限点である。

\subsection{論理語、操作、そして無限}

論理語(論理定項)が名でないことを考察しよう。

論理語が名でないならば、論理語は対象を現さない。それゆえ論理語に関わる理論は世界のあり方についての理論ではない。「ウィトゲンシュタイン」という固有名はウィトゲンシュタインという対象を表す。それゆえ、そこにはウィトゲンシュタインという対象のあり方について、深く調べることができる。「ウィトゲンシュタインは第一次世界大戦に従軍した」「ウィトゲンシュタインは1951年62歳で死んだ」など、「ウィトゲンシュタイン」という名を用いた真なる情報を集めれば「ウィトゲンシュタイン」について分析できる。\footnote{野矢先生の本の本文では知識を形成するとある}

一方で、論理語はどうかというとそうではない。「よって」「または」などの真なる命題とはそもそも何なのか規定しうることも叶わない。したがって、論理語は名でないと結論を出すことができる。

何より、「操作」という概念自体が「無限」に深く関わっている。

\begin{quote}
  5.2523 | 操作の反復適用という概念は「以下同様」という概念に等しい。
\end{quote}

「以下同様」の「以下」に無限が隠れていることがわかるだろうか。「以下」には範囲が規定されていないが、これからさきもずっと「同様」であることがひと目でわかるだろう。 これが「無限」なのである。

論考の目標は思考の限界を捉えることである。しかし、思考の限界を思考することこそ不可能である。だからこそ、思考の限界を画定するために「無限」を用いないといけない局面が必ず訪れるに違いない。

\begin{thebibliography}{9}
  \bibitem{1} 論理哲i学論考,ウィトゲンシュタイン(訳:野矢茂樹),岩波文庫,2003
  \bibitem{2} 論理哲学論考,ウィトゲンシュタイン(訳:丘沢静也),光文社古典新訳文庫,2014
  \bibitem{3} ウィトゲンシュタイン『論理哲学論考』を読む,野矢茂樹,ちくま学芸文庫,2002
\end{thebibliography}


\end{document}
