\documentclass[12pt]{jsarticle}
\usepackage[top=20truemm,bottom=20truemm,left=25truemm,right=25truemm]{geometry}
\usepackage[dvipdfmx]{graphicx}
\usepackage{float}
\usepackage{amsmath,amssymb}
\begin{document}
\newcommand{\bvec}[1]{\mbox{\boldmath $#1$}}
\begin{flushright}
2016/02/27(Sun) 山川 暁久
\end{flushright} \\ 
\begin{LARGE}
東工大文芸部『論理哲学論考』読書会ファイナル
\end{LARGE}
\begin{flushright}
-論考の向こう側へ-
\end{flushright}

\section{導入}
\subsection*{範囲}
6.2-7\footnote{今回は光文社古典新訳文庫の論考より引用します。いちおう理由はあるので、それについては後述します。}\

\subsection*{概要} 
6.2-6.3751 : 論理と数学・自然科学について。\par
6.4-7 : 倫理と世界の外側、そして語り得ないもの。 

\subsection*{進め方}
\begin{enumerate}
\item これまでの内容の復習
\item 該当範囲の読み合わせ\footnote{前回の読書会で提示された課題が分からなかったので、前回からしばらく間が空いたことも鑑みて、前回の内容を少し詳しく確認していくことで、前回の疑問点への回答に代えさせてもらうことにします。}\
\item レジュメ読み合わせ
\item まとめ
\item 今後の活動についての話し合い
\end{enumerate}

\subsection*{これまでのあらすじ}
論考を構成する3本の柱を挙げるとすれば、例えば以下のようなものでしょう。
\begin{enumerate}
\item 像の理論 (e.g. 2.1-2.225)
\item 真理関数の理論 (e.g. 5.32, 6)
\item 「語る」と「示す」の区別 (e.g. 4.12, 6.12)
\end{enumerate}
\par
前回(5.55-6.13)は、2,3と結びつきのある以下の二点が主題でした。
\begin{quote}
5.6 私の言語の限界は、私の世界の限界を意味する。
\end{quote}
\begin{quote}
6.12 論理学の命題はトートロジーである。\\
これは、言語の、つまり世界の、形式的なー論理的なー特性を示している。
\end{quote}
\par
5.6は、"私の論理空間内部が語り得ずただ示されうること”と、"私の論理空間外部を語ることの不可能性”という二点と密接に関わり合っています。これは、論考で述べられている独我論(言語論的独我論)の特徴でもあります。論考の独我論が現象学的ではなく、いわば言語論的であることは6.12から明らかです。 
\par
6.12は、ある命題がトートロジーであることから論理の形式が示される、ということ主張しています。すなわち、論理語をある仕方で組み合わせることで真理領域が論理空間全体に及ぶ命題が構成できて、このことから、そこで用いられている真理操作の内実を明確にすることができる、ということです。 
\par
トートロジーは経験的内容を持たないため、ある命題がトートロジーであるということから世界の普遍的枠組みを形式的に示すことができます。一方で、言語は経験的内容を含み、個別的なものです。ここから独我論が立ち上がってくることになります。 

\subsection*{今回やること}
独我論と真理関数の議論で論考の大きな山は越えたことになります。現時点を例えるなら、建物の階段を駆け上がって屋上にたどり着き、ようやく眺めを一望できたところでしょう。今回は、屋上からの眺めをもう少し楽しんでから、取水塔にハシゴを伝って登り、そのハシゴを投げ捨てることになります。
\par
前回扱ってきた論理学の命題に関する議論は数学・自然科学においても地続きになっていて、同様の議論を行うことができます。論理学では、ある命題がトートロジーであることから論理学の有する形式が示されました。数学・自然科学については、論理学におけるトートロジーが等式・因果法則に対応しています。
\par
そして今回の主題となるのは後半の倫理に関する議論(6.4以降)です。ここから論考の趣がガラッと変化し、世界の外側へ、さらには語り得ないものへの議論へ至ることになります。

\section{今回の範囲}
今回の範囲では、これまでのような論理記号を用いた議論はほとんど登場しませんが、深い洞察を含むものがほとんどです。分量自体はそれほど多くないので、一つ一つのセンテンスを、特に後半部分について、追っていくことにします。

\subsection{論理学と数学・自然科学}
前回、論理学の命題がトートロジーに帰着されることを見てきました。トートロジーとは、論理空間全域で常に真であるような命題のことです。論理命題には、それを構成する要素命題(基底)と、基底に対する論理操作が存在しました。基底は経験的であり、個々人の持つ言語に依存します。一方で、操作はアプリオリなものです。
\par
論理命題がトートロジーであるということは、論理命題の真理性が基底に全く依存していない、ということです。それゆえ、論理命題がトートロジーであることが、論理操作のあり方を示している、ということになります。この事実から以下のような言明がなされます。
\begin{quote}
6.121 論理学の命題は、命題を、なにも言っていない命題に結合することによって命題の論理的な特性を具体的に記述する。
\end{quote}
\begin{quote}
6.1261 論理ではプロセスと結果は等価である。\\(そういうわけだから、驚くようなことはない)
\end{quote}
\par
論理命題は、世界記述の命題ではありませんが、私たちが世界記述する際に用いる真理操作のあり方、ひいては記述の形式を示しています。このことは、論考中において以下のように述べられています。
\begin{quote}
6.124 (...)論理学の命題は、なにも「扱って」いない。論理学の命題が前提にしているのは、名前には指示対象があり、要素命題には意味があるということである。そしてこのことによって、論理学の命題は世界と結びついているのだ。(...)
\end{quote}
\begin{quote}
6.13 論理は学説ではない。世界の鏡像である。論理は、超越論的である。
\end{quote}
\par
論考では、ここからさらに一歩進んで、数学・自然科学についてこの議論を援用していきます。すなわち、数学や自然科学の命題もまた、世界記述ではなく、記述の形式を示している、ということです。
\par
まず数学についてです。論理の命題がトートロジーであるように、数学の命題は等式です。ゆえに、ここから記述の形式が示されることになります。
\begin{quote}
6.22 世界の論理を、論理の命題はトートロジーにおいてしめすのだが、数学は等式においてしめす。
\end{quote}
\par
"論理命題がなにも「扱って」いない"といわれたように、数学の命題についても等式で結ばれた2つの表現が示していることは、"2つの指示対象が同じである"ということです。しかし、2つの表現の指示対象が一致していることを\textbf{主張する}ことはできません。それはただ示されるだけです。\\
例えば、$\sim\sim(\sim p)=\sim\sim\sim p$という等式を考えると、これを操作の数という観点から述べたものは$1+2=3$に対応します。このとき、既に私たちは等式の右辺と左辺の表現をよく知っており、この2つの指示対象が同じであるという立場から、観察していることになります。
\par
さらに、私たちが数学の問題を解くために、必要な直観は言語が配達してくれます。言語は経験に依存するからです。たとえば、子どもに数学を教える状況を考えてみましょう。小学校入りたての子どもであれば、$1+2=3$のような足し算もおぼつかないでしょう。しかし、この2つの表現を子どもに「1に2を加えたものは3です。」と言語によって再構成してあげることで、この等式が成り立つ直観
を与えることにつながります。ただし人それぞれ異なる論理空間を持つため、直観が完全に受け取られるわけではなく"配達する"ことになります。しかし、このようなプロセスがなければ、この等式が成立することを理解することは不可能でしょう。\par
この議論は、論理についてもあてはまります。論理を共有しない人へ論理語の操作のあり方を説明するのにトートロジーを使おうとしても、理解され得ない状況が生み出されるのと同じです。数学においては、計算のプロセスが、まさにその直観を促してくれます。
\par
一方、自然科学の法則命題もまた記述の形式を与えるものであると論じています。これは端的に、以下のように述べてあります。

\begin{quote}
6.343 力学は、世界の記述に私たちが必要とする、真の命題のすべてを、たった1つのプランにそって構成しようとする試みである。
\end{quote}
\begin{quote}
6.36 もし仮に因果法則というものが存在するなら、それは、「自然法則が存在する」というものになるかもしれない。\\
しかしもちろん、それを言うことはできない。それはしめされるのだ。
\end{quote}

力学(あるいは自然科学)は、ある公理系のもとで、世界の一般的な記述を与えるものです。科学の命題というのは、科学の命題に与えることのできる形式をアプリオリに洞察したものであり、この形式を免れるものは記述され得ません。この点において、自然科学は、昔の人々にとっての神や運命といったものと同等の位置にあります。一方で、論理には必然性があります。ここから自然科学と論理学の立脚点の相違が見えてきます。
\begin{quote}
6.37 あのことが起きたから、このことが起きるに違いないだろう、という強制は存在しない。存在するのは、論理の必然だけだ。
\end{quote}
\par
このようにして、再び論理が超越論的であることが強調されました。これは世界の形式であるがゆえに語り得ず、超越論的であるということです。これは、先験的と読み替えることもできます。\par
ここまでで一通り、『論考』の言語論は完成したことになります。すなわち、真理関数の議論を経て、要素命題について言われていた「言語のなかに映っているものを、言語は描くことができない(4.121)」という主張が、言語全般に拡張され、一般化されたということです。

\subsection{倫理}
さて、論考の趣がガラッと変わるのは6.4からです。ここからは視線が世界の外側へ向けられ、これまで議論してきた「論理」からかけ離れた「倫理・美・死・神秘」といったものが主題となってきます。これらは主体の「生」に関わる問題です。6.4は以下のようなものです。
\begin{quote}
6.4 すべての命題は等価値である。
\end{quote}
続く6.41では「世界のなかには価値は存在しない」と主張されます。すなわち、等価値とは価値が存在しないことを意味します。世界は事実の総体であると冒頭で述べられたように、世界のなかで、あるようにしてあるものは全て偶然的な事実です。それらを偶然でないものとするのが、価値の働きです。しかし、この価値は、さきほどの事情から世界の外側になければならず、世界の外側にあるものについて語ることはできません。\par
世界の偶然的事実についての記述を与えるのが自然科学の命題だとすれば、世界のあるべき形を指定する価値について語る命題は、倫理の範疇になります。このことから「倫理は超越論的である。(6.422)」と主張されます。\par
倫理は善・悪に関する言説です。倫理によって世界の境界が変容することは6.43で述べられている通りです。すなわち、善意・悪意といった倫理的問題が変えるのは、世界の限界であり、その内容(事実)を変えるのでもなければ、形式(論理形式)を変えるのでもありません。世界は、いわば総体としてその実質を変えるということです。ゆえに、世界と人生の価値は、世界の外側にあるとはいえ、世界を形作るという仕方で、世界そのものとしてあることになります。
\par
倫理によって世界の外側から私の世界が形成されること。ここでの議論によって死や神についても説明がつきます。死は人生の出来事ではありません。また、死を人は経験することはありません。世界という画面を眺める主体が居なくなることは、世界が変わったことを意味するのではなく世界が終わったことを意味する、ということです。神は世界のなかに自らを啓示することはありません。また、思考可能でないものによって世界が形成されているという気配のことをウィトゲンシュタインは神秘と呼んでいます。口に出せないにも関わらず、自分を示すものとして、神秘が存在するということです。
\par
私たちは世界の境界に迫る倫理的な問題、いわば「生」の問題について私たちの世界の内部で解決することはできません。世界の内部で解決可能なのは自然科学に関する問題なので、たとえ考えられうる全ての科学の問いに答えられようとも、私たちの生の問題には、まったく触れられていない、ということになります。

\subsection{謎の解消、そして語り得ないもの}
解かれるべきは自然科学の問題ではない一方で、その倫理的問題を語ることはできない。この身動きの取れなさは6.5の主張がなされることにより、見かけ上解消されることになります。
\begin{quote}
6.5 口にすることができない答えにたいしては、その問いも口にすることができない。\\
謎は存在しない。\\
そもそも問うことができるなら、その問いには答えることもできる。
\end{quote}
しかし、ここでの「謎は存在しない」というのは方便でしかありません。
\begin{quote}
6.54 私の文章は、つぎのような仕掛けで説明をしている。私がここで書いていることを理解する人は、私の文章を通り--私の文章に乗り--私の文章を超えて上ってしまってから、最後に、私の文章がノンセンスであることに気付くのである。(いわば、ハシゴを上ってしまったら、そのハシゴを投げ捨てるにちがいない)\\
その人は、これらの文章を克服するに違いない。そうすれば世界を正しく見ることになる。
\end{quote}
内側から境界線を引き終えたとき、そこで哲学の活動も停止することになります。その境界線上に立ち尽くしたまま、私たちはこうつぶやくほかありません。
\\
\begin{quote}
7 語ることができないことについては、沈黙するしかない。
\end{quote}
\newpage

%\section{まとめ}

\section{雑感}
倫理に関する議論は、\cite{光文社} の解説を援用している。この部分によっては最後の議論の終着点がどこに向かうのかが少し変わってくる。論考自体が自己否定的な書物であるという大筋は揺るがない。
\par
例えば、\cite{入門}では"「言語のうちに映し出されるものを、言語が描き出すことができない(4.121)」というまさにその主張のみを論考は語っている"というメタ言語的な主張がなされるが、これは少し微妙な気がする。なぜならそれはしめされうるものであり、「しめされうるものは、言われ得ない。(4.1212)」のだから。また、\cite{読む, 岩波}については「草稿」を基にした議論がなされており、ウィトゲンシュタインの肉声を聞いているかのような力強い文章でつづられている。\cite{読む, 岩波}でポイントとなるのは、私の意志によって「生」を特徴づける点である。これまで述べられた論考の言語論の下で、世界の事実をありのままに受け取る主体には幸福も不幸もない。世界の中に価値は存在しないからである。この無機質な世界は、生きる意志でもってカラフルに色づくことになる。この世界はもちろん事実の総体であり、私の論理空間の下にある。
\par
しかし、論考という書物は、「草稿」における"幸福に生きよ"というウィトゲンシュタインの声が最後に響き渡るほど楽観的なものだとは思えない。最後のセンテンスにおける沈黙は、深い世界の沈黙であるのだから。このような事情を考慮すると、「謎は存在しない」からラストへの議論が、思考の境界線上で立ち止まっているように見える\cite{光文社}の論考が一番しっくりくる。
\par
いずれにせよ、論考の最終部分の議論で重要なのは、"倫理が世界の条件である"ことである。最後に「草稿」におけるウィトゲンシュタインの声を聞いて、この章を締めることにする。

\begin{quote}
善と悪は主体によってはじめて登場する。そして主体は世界に属さない。それは世界の限界である。\\
(...)主体が世界の一部ではなく世界の前提であるように、善悪は世界の中の性質ではなく、主体の述語なのだ。\\
主体の本質はまだまったくベールの向こうにある。\\
そうだ。私の仕事は論理の基礎から世界の本質へと広がってきている。
\end{quote}

\newpage
\begin{thebibliography}{9}
\bibitem{光文社}
  ヴィトゲンシュタイン(丘沢静也 訳)「論理哲学論考」 光文社古典新訳文庫(光文社, 2014)
\bibitem{読む}
  野矢茂樹 「ウィトゲンシュタイン『論理哲学論考』を読む」 ちくま学芸文庫 (筑摩書房, 2006)
\bibitem{岩波}
  ウィトゲンシュタイン(野矢茂樹 訳)「論理哲学論考」 岩波文庫 青689-1 (岩波書店, 2003)
\bibitem{入門}
  永井均「ウィトゲンシュタイン入門」 ちくま新書 020 (筑摩書房, 1995)
\end{thebibliography}

\end{document}