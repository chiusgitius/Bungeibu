\documentclass[onecolumn,11pt]{jsbook}
\usepackage{url}
\usepackage[dvipdfmx]{hyperref}
\usepackage{pxjahyper}
\hypersetup{% hyperrefオプションリスト
setpagesize=false,
 bookmarksnumbered=true,%
 bookmarksopen=true,%
 colorlinks=true,%
 linkcolor=blue,
 citecolor=red,
}

\begin{document}

\chapter{サブタイトル}

著者:

\section{範囲}

%ここに執筆する範囲を書く。

\section{はじめに}

\section{問題}

%ここに読書会で出た発展課題を「読者にも読んで理解できるように」書く。

\section{問題に対する考察}

%哲学に終わりはないので「解答」という書き方はふさわしくないと思った。
%ここに問題に対する考察を書く。

\begin{thebibliography}{9}
  \bibitem{キー1} 参考文献の名前・著者1
  \bibitem{キー2} 参考文献の名前・著者2

  ・・・

  \bibitem{キーN} 参考文献の名前・著者2
\end{thebibliography}

\end{document}
